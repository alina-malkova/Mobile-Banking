\documentclass[12pt]{article}

% Packages
\usepackage[utf8]{inputenc}
\usepackage[margin=1in]{geometry}
\usepackage{amsmath,amssymb,amsthm}
\usepackage{graphicx}
\usepackage{booktabs}
\usepackage{longtable}
\usepackage{setspace}
\usepackage{natbib}
\usepackage{hyperref}
\usepackage{caption}
\usepackage{subcaption}
\usepackage{float}
\usepackage{threeparttable}
\usepackage{multirow}
\usepackage{pdflscape}

% Double spacing
\doublespacing

% Title
\title{Mobile Banking, Bank Branch Closures, and Self-Employment in the United States}

\author{Alina Malkova\thanks{Florida Institute of Technology. Email: amalkova@fit.edu}}

\date{February 2026 \\ \textit{Preliminary Draft --- Please Do Not Cite}}

\begin{document}

\maketitle

\begin{abstract}
\noindent This paper investigates whether mobile banking can substitute for traditional branch banking in supporting self-employment. Using data from the FDIC National Survey of Unbanked and Underbanked Households (2013--2023), I document that self-employment rates are significantly higher among branch banking users (9.95\%) compared to mobile-only users (7.19\%). I develop a static structural model of joint banking mode and employment choice with unobserved heterogeneity, correctly specified for the repeated cross-sectional data structure. Following recent econometric advances, I employ a three-pronged approach to select the number of latent types: Hao-Kasahara (2025) panel BIC, Bonhomme-Lamadon-Manresa (2022) counterfactual stability, and Budanova (2025) penalized MLE. All three methods support $K=4$ types with heterogeneous responses to branch closures, including a substantial group (32\%) highly dependent on branch-based relationship lending. Counterfactual simulations show that a 50\% reduction in branch access reduces aggregate self-employment by approximately 11\%, with meaningful sensitivity to unobserved heterogeneity assumptions (effects range from 1\% to 11\% across specifications). The results demonstrate the importance of proper model selection in mixture models for policy evaluation.

\vspace{0.5cm}
\noindent \textbf{JEL Codes:} G21, J24, L26, O33, R12

\vspace{0.3cm}
\noindent \textbf{Keywords:} Mobile banking, self-employment, entrepreneurship, bank branches, financial inclusion, banking deserts
\end{abstract}

\newpage

%%%%%%%%%%%%%%%%%%%%%%%%%%%%%%%%%%%%%%%%%%%%%%%%%%%%%%%%%%%%%%%%%%%%%%%%%%%%%%%
\section{Introduction}
%%%%%%%%%%%%%%%%%%%%%%%%%%%%%%%%%%%%%%%%%%%%%%%%%%%%%%%%%%%%%%%%%%%%%%%%%%%%%%%

The landscape of retail banking in the United States has undergone a dramatic transformation over the past decade. Between 2019 and 2023, U.S. bank branches declined by 5.6\%, with over 4,000 branch closures nationwide \citep{fdic2023}. Simultaneously, mobile banking adoption has surged: the share of banked households reporting mobile banking as their primary account access method rose from 15.1\% in 2017 to 43.5\% in 2021 \citep{fdic2021}. These trends have left approximately 12.3 million Americans living in ``banking deserts''---communities lacking physical bank branches within reasonable distance.

This paper investigates a critical question at the intersection of financial access and entrepreneurship: Does mobile banking serve as a substitute channel for credit access that supports self-employment in areas with declining branch presence? Self-employment represents a vital pathway to economic mobility, particularly for populations historically underserved by traditional financial institutions. If mobile banking can effectively replace branch-based banking relationships for entrepreneurial credit access, the ongoing digitization of financial services may partially offset the negative effects of branch closures on local economic dynamism. Conversely, if branch relationships remain essential for accessing the credit and financial services that enable entrepreneurship, the geographic concentration of branch closures in lower-income and minority communities may exacerbate existing disparities in entrepreneurship rates.

I study this question using microdata from the FDIC National Survey of Unbanked and Underbanked Households, which is administered biennially as a supplement to the Current Population Survey. The survey provides detailed information on banking behaviors, including the specific channels households use to access their accounts, combined with employment status from the CPS base survey. Importantly, the CPS identifies self-employment through its class-of-worker variable, allowing me to distinguish between wage employment and entrepreneurship.

The empirical analysis proceeds in two stages. First, I document descriptive patterns and estimate reduced-form relationships between mobile banking adoption and self-employment. The raw data reveal a striking pattern: households that primarily use branch banking have a self-employment rate of 12.6\%, compared to just 8.7\% among mobile-only banking users. However, this correlation likely reflects selection---the same characteristics that lead individuals to prefer branch banking (older age, higher wealth, established business relationships) may also be associated with higher rates of self-employment. After controlling for demographics, education, income, and CBSA fixed effects, the relationship between mobile banking and self-employment becomes small and statistically insignificant. Instrumental variable estimates using local broadband penetration as an instrument for mobile banking adoption yield positive but imprecise effects.

Second, I develop a structural model of joint banking mode and employment status choice. Individuals choose from three banking modes (unbanked, mobile/online only, branch user) and three employment statuses (wage employment, self-employment, not working), yielding nine discrete choice alternatives. The key structural parameters capture how banking mode affects access to credit, and how credit access in turn affects the returns to self-employment. This framework allows me to decompose the observed correlation between branch banking and self-employment into (i) selection effects (who chooses each banking mode), (ii) direct effects (how banking mode affects employment outcomes), and (iii) the role of local banking infrastructure in shaping both choices.

The structural model enables counterfactual policy analysis that reduced-form methods cannot provide. Specifically, I can simulate the effects of: (1) continued branch closures with no change in mobile banking access; (2) branch closures accompanied by improvements in broadband infrastructure that facilitate mobile banking adoption; and (3) targeted subsidies for mobile banking adoption in banking deserts.

This paper contributes to several literatures. First, it adds to the growing body of work on the real effects of bank branch closures \citep{nguyen2019, granja2022, celerier2019}. While existing research has documented effects on small business lending and local economic activity, I provide the first evidence specifically on self-employment entry. Second, the paper contributes to the literature on technology and financial inclusion \citep{jack2014, muralidharan2016, breza2020}, extending the analysis from developing country contexts to examine whether mobile technology can substitute for physical banking infrastructure in advanced economies. Third, I contribute methodologically by developing a structural framework for analyzing the joint determination of banking mode and employment status, which can be applied to study other aspects of financial access and labor market outcomes.

%%%%%%%%%%%%%%%%%%%%%%%%%%%%%%%%%%%%%%%%%%%%%%%%%%%%%%%%%%%%%%%%%%%%%%%%%%%%%%%
\section{Background and Institutional Context}
%%%%%%%%%%%%%%%%%%%%%%%%%%%%%%%%%%%%%%%%%%%%%%%%%%%%%%%%%%%%%%%%%%%%%%%%%%%%%%%

\subsection{Bank Branch Closures in the United States}

The consolidation of the U.S. banking sector has accelerated in recent years. Following the 2008 financial crisis, regulatory changes increased compliance costs for small banks, spurring mergers and branch network optimization. More recently, the COVID-19 pandemic accelerated the shift toward digital banking, leading banks to close branches deemed redundant.

Branch closures have not been geographically uniform. Rural areas, low-income urban neighborhoods, and communities with higher shares of minority residents have experienced disproportionate declines in branch presence \citep{morgan2016, ergungor2010}. This pattern raises concerns about equitable access to financial services, as branch relationships remain important for accessing certain products---particularly small business credit that relies on soft information and relationship lending \citep{petersen2002, berger2005}.

\subsection{Mobile Banking Adoption}

Mobile banking technology has evolved rapidly from simple balance checking to comprehensive financial management platforms. Modern mobile banking applications allow users to deposit checks, transfer funds, apply for loans, and manage investments. The Federal Reserve's survey of household financial technology use documents steady increases in mobile banking adoption across all demographic groups, though significant disparities remain by age, income, and education \citep{fed2022}.

For entrepreneurs and self-employed individuals, mobile banking offers potential benefits including: reduced transaction costs for managing business finances, faster access to account information for cash flow management, and the ability to conduct banking outside traditional business hours. However, mobile banking may be less effective than branch relationships for establishing the trust and soft information transmission that facilitate access to credit.

\subsection{Self-Employment and Credit Access}

Self-employment requires access to capital for startup costs, working capital, and investment in growth. Traditional bank lending to small businesses relies heavily on relationship banking, where loan officers develop knowledge about borrowers through repeated interactions \citep{berger1995}. This model inherently favors borrowers with physical access to branches.

Recent research has examined alternative financing channels for entrepreneurs, including online lending platforms \citep{morse2015, tang2019}, fintech credit scoring \citep{berg2020, fuster2019}, and mobile money in developing countries \citep{beck2018}. However, evidence on whether mobile banking---as distinct from mobile lending---affects entrepreneurship in advanced economies remains limited.

%%%%%%%%%%%%%%%%%%%%%%%%%%%%%%%%%%%%%%%%%%%%%%%%%%%%%%%%%%%%%%%%%%%%%%%%%%%%%%%
\section{Data}
%%%%%%%%%%%%%%%%%%%%%%%%%%%%%%%%%%%%%%%%%%%%%%%%%%%%%%%%%%%%%%%%%%%%%%%%%%%%%%%

\subsection{FDIC National Survey of Unbanked and Underbanked Households}

The primary data source is the FDIC National Survey of Unbanked and Underbanked Households, conducted biennially since 2009 as a supplement to the June Current Population Survey. The survey collects detailed information on household banking status, account types, methods of accessing accounts, and use of alternative financial services.

For this analysis, I use the multi-year public use microdata file covering survey waves from 2009 to 2023, yielding approximately 570,000 household-level observations. The survey includes harmonized variables across waves, enabling consistent measurement of banking behaviors over time. Key variables from the FDIC supplement include:

\begin{itemize}
    \item \textbf{Banking status:} Whether the household has a bank account (checking, savings, or both), and detailed underbanking measures based on use of alternative financial services.

    \item \textbf{Account access methods:} The specific channels used to access bank accounts, including branch visits, ATM, telephone, online banking, and mobile banking. Crucially, the survey asks which method is used most frequently.

    \item \textbf{Mobile banking activities:} For mobile banking users, detailed information on specific activities conducted (balance checking, bill payment, deposits, transfers, etc.).
\end{itemize}

\subsection{Current Population Survey}

Because the FDIC survey is administered as a CPS supplement, I observe the full set of CPS variables for each respondent. Key variables from the CPS base survey include:

\begin{itemize}
    \item \textbf{Employment status:} Labor force participation, employment/unemployment, and class of worker (wage and salary vs. self-employed, with distinction between incorporated and unincorporated self-employment).

    \item \textbf{Demographics:} Age, sex, race/ethnicity, education, marital status, and household composition.

    \item \textbf{Geography:} State, Core-Based Statistical Area (CBSA), and metropolitan status. Geographic identifiers enable merging with area-level data on banking infrastructure and economic conditions.
\end{itemize}

\subsection{FDIC Summary of Deposits}

I supplement the survey data with information on local banking infrastructure from the FDIC Summary of Deposits (SOD), which provides an annual census of all FDIC-insured bank branches including their precise locations. From the SOD, I construct CBSA-year measures of:

\begin{itemize}
    \item Total number of bank branches
    \item Branch density (branches per capita or per square mile)
    \item Net branch changes (openings minus closures)
    \item Banking desert indicators (absence of branches within specified distance)
\end{itemize}

\subsection{American Community Survey}

I merge CBSA-level control variables from the American Community Survey (ACS), including:

\begin{itemize}
    \item \textbf{Broadband penetration:} Share of households with broadband internet subscription and/or smartphone data plans. This serves as both a control variable and instrumental variable for mobile banking adoption.

    \item \textbf{Demographic composition:} Population, racial/ethnic composition, age distribution, and educational attainment.

    \item \textbf{Economic conditions:} Median household income, unemployment rate, and industry employment shares.
\end{itemize}

\subsection{Sample Construction}

The analysis sample is constructed as follows:

\begin{enumerate}
    \item Start with the FDIC multi-year microdata (N = 570,943 observations).

    \item Restrict to working-age adults (18--64) in the labor force (employed or actively seeking work), reducing the sample to observations where self-employment is a feasible choice.

    \item Keep survey waves from 2013 onward, when mobile banking questions were consistently available.

    \item Retain observations with identifiable CBSA codes for geographic analysis.
\end{enumerate}

The final analysis sample contains 125,017 individual observations across 293 CBSAs and 6 survey waves (2013, 2015, 2017, 2019, 2021, 2023).

\subsection{Variable Definitions}

\subsubsection{Banking Mode}

I classify households into three mutually exclusive banking modes:

\begin{enumerate}
    \item \textbf{Unbanked:} Household does not have a checking or savings account at a bank or credit union.

    \item \textbf{Mobile/Online Only:} Banked household that accesses accounts exclusively through mobile or online channels, without visiting bank tellers.

    \item \textbf{Branch User:} Banked household that uses bank teller services, either exclusively or in combination with other access methods.
\end{enumerate}

This classification captures the key distinction between households that maintain relationships with physical branches versus those relying entirely on digital channels.

\subsubsection{Employment Status}

Employment status is classified into three categories:

\begin{enumerate}
    \item \textbf{Wage Worker:} Employed in a wage and salary position (private sector, government, or nonprofit).

    \item \textbf{Self-Employed:} Employed in own business, either incorporated or unincorporated.

    \item \textbf{Not Working:} Unemployed (actively seeking work) or temporarily not working.
\end{enumerate}

\subsubsection{Key Control Variables}

\begin{itemize}
    \item \textbf{Age:} Continuous measure in years, with quadratic term to capture nonlinear lifecycle patterns.

    \item \textbf{Education:} Four categories: less than high school, high school diploma, some college, and college degree or higher.

    \item \textbf{Race/Ethnicity:} Seven categories following Census definitions, with separate indicators for Black, Hispanic, Asian, and White non-Hispanic.

    \item \textbf{Family Income:} Five categories: below \$15,000; \$15,000--\$30,000; \$30,000--\$50,000; \$50,000--\$75,000; and above \$75,000.

    \item \textbf{Metropolitan Status:} Indicator for residence in a metropolitan area.
\end{itemize}

%%%%%%%%%%%%%%%%%%%%%%%%%%%%%%%%%%%%%%%%%%%%%%%%%%%%%%%%%%%%%%%%%%%%%%%%%%%%%%%
\section{Descriptive Analysis}
%%%%%%%%%%%%%%%%%%%%%%%%%%%%%%%%%%%%%%%%%%%%%%%%%%%%%%%%%%%%%%%%%%%%%%%%%%%%%%%

\subsection{Trends in Mobile Banking Adoption}

Table \ref{tab:mobile_trends} documents the rise of mobile banking over the sample period. Mobile banking as the primary account access method increased from approximately 15\% in 2013 to over 40\% by 2023. This increase occurred across all demographic groups, though adoption remains higher among younger, more educated, and higher-income households.

\begin{table}[H]
\centering
\caption{Mobile Banking Adoption Trends}
\label{tab:mobile_trends}
\begin{threeparttable}
\begin{tabular}{lccccccc}
\toprule
& 2013 & 2015 & 2017 & 2019 & 2021 & 2023 \\
\midrule
Mobile banking user (\%) & 19.2 & 24.8 & 29.1 & 31.5 & 38.4 & 42.1 \\
Mobile as primary (\%) & 8.3 & 11.2 & 13.8 & 15.7 & 21.3 & 25.6 \\
Branch user (\%) & 78.4 & 74.1 & 70.2 & 68.3 & 61.2 & 57.8 \\
Unbanked (\%) & 7.2 & 6.8 & 6.5 & 5.4 & 4.8 & 4.5 \\
\midrule
N & 21,105 & 21,892 & 21,456 & 20,127 & 19,234 & 21,203 \\
\bottomrule
\end{tabular}
\begin{tablenotes}
\small
\item Notes: Sample restricted to working-age adults (18--64) in the labor force. Statistics are weighted using survey weights.
\end{tablenotes}
\end{threeparttable}
\end{table}

\subsection{Self-Employment by Banking Mode}

Table \ref{tab:se_by_banking} presents self-employment rates by banking mode. The key finding is that branch users have substantially higher self-employment rates (12.6\%) compared to mobile-only users (8.7\%) and unbanked households (10.1\%).

\begin{table}[H]
\centering
\caption{Self-Employment Rates by Banking Mode}
\label{tab:se_by_banking}
\begin{threeparttable}
\begin{tabular}{lccc}
\toprule
Banking Mode & Self-Employment Rate & Std. Error & N \\
\midrule
Unbanked & 10.06\% & (0.52) & 31,929 \\
Mobile/Online Only & 8.70\% & (0.48) & 6,526 \\
Branch User & 12.63\% & (0.21) & 60,328 \\
\midrule
All & 11.06\% & (0.18) & 98,783 \\
\bottomrule
\end{tabular}
\begin{tablenotes}
\small
\item Notes: Self-employment includes both incorporated and unincorporated self-employment. Statistics are weighted using survey weights. Standard errors in parentheses.
\end{tablenotes}
\end{threeparttable}
\end{table}

\subsection{Joint Distribution of Banking and Employment}

Table \ref{tab:joint_dist} presents the full joint distribution of banking mode and employment status. The dominant category is Branch User + Wage Worker (71.9\%), followed by Branch User + Self-Employed (9.0\%). Mobile/Online users are predominantly wage workers (8.7\%) with a small share of self-employed (0.7\%).

\begin{table}[H]
\centering
\caption{Joint Distribution of Banking Mode and Employment Status}
\label{tab:joint_dist}
\begin{threeparttable}
\begin{tabular}{lcccc}
\toprule
& Wage Worker & Self-Employed & Not Working & Total \\
\midrule
Unbanked & 4.50\% & 0.53\% & 1.11\% & 6.14\% \\
Mobile/Online Only & 8.66\% & 0.71\% & 0.32\% & 9.69\% \\
Branch User & 71.90\% & 9.03\% & 3.24\% & 84.17\% \\
\midrule
Total & 85.06\% & 10.27\% & 4.67\% & 100.00\% \\
\bottomrule
\end{tabular}
\begin{tablenotes}
\small
\item Notes: Sample restricted to observations with non-missing banking mode. Statistics weighted using survey weights.
\end{tablenotes}
\end{threeparttable}
\end{table}

%%%%%%%%%%%%%%%%%%%%%%%%%%%%%%%%%%%%%%%%%%%%%%%%%%%%%%%%%%%%%%%%%%%%%%%%%%%%%%%
\section{Reduced-Form Evidence}
\label{sec:reduced_form}
%%%%%%%%%%%%%%%%%%%%%%%%%%%%%%%%%%%%%%%%%%%%%%%%%%%%%%%%%%%%%%%%%%%%%%%%%%%%%%%

\subsection{Baseline OLS Specifications}

I estimate the following baseline specification:

\begin{equation}
SE_{ijt} = \alpha + \beta \cdot MobileBanking_{ijt} + X_{ijt}'\gamma + \phi_j + \lambda_t + \varepsilon_{ijt}
\label{eq:baseline}
\end{equation}

where $SE_{ijt}$ is an indicator for self-employment for individual $i$ in CBSA $j$ at time $t$, $MobileBanking_{ijt}$ is an indicator for mobile banking use, $X_{ijt}$ is a vector of individual controls, $\phi_j$ are CBSA fixed effects, and $\lambda_t$ are year fixed effects.

Table \ref{tab:baseline_ols} presents the results. Column (1) shows the raw correlation: mobile banking users are 1.24 percentage points less likely to be self-employed. This negative correlation is reversed after adding demographic controls (Column 2) and becomes small and statistically insignificant with CBSA and year fixed effects (Columns 3--4).

\begin{table}[H]
\centering
\caption{Baseline OLS: Self-Employment and Mobile Banking}
\label{tab:baseline_ols}
\begin{threeparttable}
\begin{tabular}{lcccc}
\toprule
& (1) & (2) & (3) & (4) \\
\midrule
Mobile Banking User & $-0.0124^{***}$ & 0.0051 & 0.0036 & 0.0034 \\
& (0.0037) & (0.0035) & (0.0037) & (0.0037) \\
\\
Age & & $0.0054^{***}$ & $0.0053^{***}$ & $0.0054^{***}$ \\
& & (0.0012) & (0.0012) & (0.0012) \\
\\
\midrule
Demographics & No & Yes & Yes & Yes \\
CBSA FE & No & No & Yes & Yes \\
Year FE & No & No & Yes & Yes \\
CBSA Controls & No & No & No & Yes \\
\midrule
Observations & 45,944 & 45,944 & 45,944 & 45,466 \\
R-squared & 0.000 & 0.019 & 0.030 & 0.029 \\
\bottomrule
\end{tabular}
\begin{tablenotes}
\small
\item Notes: Dependent variable is an indicator for self-employment. Demographic controls include age, age squared, education, race/ethnicity, and family income categories. CBSA controls include broadband penetration and unemployment rate. Standard errors clustered at CBSA level in parentheses. Survey weights applied. $^{*}p<0.10$, $^{**}p<0.05$, $^{***}p<0.01$.
\end{tablenotes}
\end{threeparttable}
\end{table}

\subsection{Instrumental Variables Estimation}

To address potential endogeneity of mobile banking adoption, I instrument for mobile banking using CBSA-level broadband penetration. The identifying assumption is that broadband infrastructure affects self-employment only through its effect on mobile banking adoption, conditional on other controls.

The first-stage relationship is:
\begin{equation}
MobileBanking_{ijt} = \delta + \pi \cdot Broadband_{jt} + X_{ijt}'\theta + \mu_s + \lambda_t + \nu_{ijt}
\end{equation}

where $\mu_s$ are state fixed effects (replacing CBSA fixed effects to allow for cross-CBSA variation in broadband).

Table \ref{tab:iv_results} presents the IV results. The first stage shows that broadband penetration significantly predicts mobile banking adoption. The reduced form shows a positive relationship between broadband and self-employment. The IV estimate is positive but imprecisely estimated due to the weak first stage.

\begin{table}[H]
\centering
\caption{IV Estimates: Broadband as Instrument for Mobile Banking}
\label{tab:iv_results}
\begin{threeparttable}
\begin{tabular}{lccc}
\toprule
& First Stage & Reduced Form & IV \\
& (Mobile Banking) & (Self-Employment) & (Self-Employment) \\
\midrule
Broadband Penetration & $0.0051^{**}$ & $0.0018^{*}$ & \\
& (0.0023) & (0.0010) & \\
\\
Mobile Banking & & & 0.344 \\
& & & (0.283) \\
\midrule
State FE & Yes & Yes & Yes \\
Year FE & Yes & Yes & Yes \\
Demographics & Yes & Yes & Yes \\
\midrule
Observations & 45,506 & 86,562 & 45,466 \\
First-stage F & 4.82 & --- & --- \\
\bottomrule
\end{tabular}
\begin{tablenotes}
\small
\item Notes: IV estimation uses broadband penetration as instrument for mobile banking. State fixed effects used instead of CBSA fixed effects to allow for cross-CBSA variation in broadband. Standard errors clustered at state level. $^{*}p<0.10$, $^{**}p<0.05$, $^{***}p<0.01$.
\end{tablenotes}
\end{threeparttable}
\end{table}

\subsection{Post-Double-Selection LASSO}

A potential concern with the baseline results is that the null finding could be an artifact of functional form assumptions or model selection. To address this, I implement post-double-selection LASSO following \citet{belloni2014}. This method uses LASSO to select controls from a large candidate set (all pairwise interactions, polynomials, and CBSA characteristics) for both the outcome equation (self-employment) and the treatment equation (mobile banking), then estimates the treatment effect using the union of selected variables.

Table \ref{tab:lasso} presents the results. Column (1) replicates the hand-selected OLS specification from Table \ref{tab:baseline_ols}. Column (2) reports the post-double-selection estimate using 47 candidate controls including demographic interactions (age $\times$ race, education $\times$ age, income $\times$ education) and CBSA characteristics (broadband polynomials, broadband $\times$ demographics). The LASSO procedure selects 12 controls for the outcome equation and 8 for the treatment equation.

The PDS-LASSO coefficient (0.0031, $p = 0.42$) is nearly identical to the hand-selected OLS estimate, confirming that the null result is not driven by functional form assumptions. This strengthens confidence in the reduced-form finding that mobile banking does not significantly affect self-employment after controlling for selection.

\textbf{Extension: PDS-LASSO IV.} The weak first-stage in the IV analysis ($F = 4.82$, below Stock-Yogo thresholds) could bias 2SLS estimates. The \citet{belloni2014} framework extends to IV settings: LASSO simultaneously selects instruments from a broader candidate set (broadband $\times$ demographics, broadband polynomials, lagged broadband changes) and controls, addressing both weak instruments and control selection in a unified framework. Implementing this extension with the \texttt{pdslasso} package in Stata yields qualitatively similar results---the null finding persists---though the IV point estimate remains imprecise due to limited exogenous variation in broadband penetration.

\begin{table}[H]
\centering
\caption{Machine Learning Robustness: PDS-LASSO and Double ML}
\label{tab:lasso}
\begin{threeparttable}
\begin{tabular}{lccc}
\toprule
& (1) & (2) & (3) \\
& Hand-Selected OLS & PDS-LASSO & Double ML \\
\midrule
Mobile Banking User & 0.0034 & 0.0031 & 0.0028 \\
& (0.0037) & (0.0038) & (0.0041) \\
\midrule
Control selection & Hand-selected & LASSO & Cross-fitted ML \\
Candidate controls & 15 & 47 & 47 \\
Selected (outcome eq.) & --- & 12 & Flexible \\
Selected (treatment eq.) & --- & 8 & Flexible \\
\midrule
Observations & 45,466 & 45,466 & 45,466 \\
\bottomrule
\end{tabular}
\begin{tablenotes}
\small
\item Notes: Column (1) replicates Column (4) of Table \ref{tab:baseline_ols}. Column (2) implements post-double-selection LASSO \citep{belloni2014}. Column (3) implements double/debiased machine learning \citep{chernozhukov2018dml} with 5-fold cross-fitting, allowing flexible functional forms in both outcome and propensity score models. The DML null is stronger than OLS/LASSO because it rules out nonlinear misspecification in both equations. Standard errors clustered at CBSA level.
\end{tablenotes}
\end{threeparttable}
\end{table}

\subsection{Heterogeneity Analysis}

Table \ref{tab:heterogeneity} explores heterogeneity in the mobile banking--self-employment relationship across demographic groups. The relationship is positive and marginally significant for middle-income households (\$50,000--\$75,000), suggesting mobile banking may facilitate entrepreneurship particularly for this group.

\begin{table}[H]
\centering
\caption{Heterogeneity in Mobile Banking Effects}
\label{tab:heterogeneity}
\begin{threeparttable}
\begin{tabular}{lccccc}
\toprule
\multicolumn{6}{c}{\textit{Panel A: By Race/Ethnicity}} \\
& Black & Hispanic & White & & \\
\midrule
Mobile Banking & $-0.002$ & 0.007 & 0.003 & & \\
& (0.009) & (0.010) & (0.004) & & \\
N & 4,815 & 5,785 & 31,682 & & \\
\midrule
\multicolumn{6}{c}{\textit{Panel B: By Income}} \\
& $<$\$15K & \$15--30K & \$30--50K & \$50--75K & $>$\$75K \\
\midrule
Mobile Banking & 0.009 & 0.004 & $-0.004$ & $0.014^{*}$ & 0.002 \\
& (0.015) & (0.010) & (0.008) & (0.007) & (0.005) \\
N & 2,441 & 4,765 & 8,286 & 9,541 & 20,823 \\
\bottomrule
\end{tabular}
\begin{tablenotes}
\small
\item Notes: Each cell reports coefficient on mobile banking indicator from separate regressions with full controls and CBSA/year fixed effects. Standard errors clustered at CBSA level. $^{*}p<0.10$, $^{**}p<0.05$, $^{***}p<0.01$.
\end{tablenotes}
\end{threeparttable}
\end{table}

As a robustness check on the hand-selected subgroup analysis, I implement a data-driven heterogeneity analysis following \citet{athey2016} and \citet{chernozhukov2018}. Specifically, I estimate individual-level treatment effects using a flexible outcome model with interactions, then examine the distribution of effects and identify which covariates best predict heterogeneity (the ``best linear predictor'' of treatment effect heterogeneity). This approach discovers heterogeneity dimensions from the data rather than imposing them ex ante.

The sorted effects analysis reveals substantial heterogeneity: the 10th percentile of individual treatment effects is $-0.02$ while the 90th percentile is $+0.03$. The best linear predictor identifies age $\times$ income interactions as the primary driver of heterogeneity---effects are larger (more positive) for middle-aged, middle-income individuals who may be at the margin of entrepreneurship. This finding is consistent with Table \ref{tab:heterogeneity} Panel B but provides additional confidence that the income heterogeneity is not a statistical artifact.

%%%%%%%%%%%%%%%%%%%%%%%%%%%%%%%%%%%%%%%%%%%%%%%%%%%%%%%%%%%%%%%%%%%%%%%%%%%%%%%
\section{Structural Model}
%%%%%%%%%%%%%%%%%%%%%%%%%%%%%%%%%%%%%%%%%%%%%%%%%%%%%%%%%%%%%%%%%%%%%%%%%%%%%%%

\subsection{Model Environment}

Consider an individual $i$ in CBSA $j$ at time $t$ who makes two interrelated discrete choices:

\begin{itemize}
    \item \textbf{Banking mode} $b \in \{U, M, B\}$ (unbanked, mobile, branch)
    \item \textbf{Employment status} $d \in \{W, S, N\}$ (wage, self-employed, not working)
\end{itemize}

This yields nine joint alternatives. The structural model identifies how credit access through different banking channels affects the returns to self-employment, and how this relationship varies across latent types.

\textbf{A note on dynamics.} In principle, banking and employment choices have dynamic elements: switching costs create persistence in banking mode, and self-employment experience accumulates over time. However, dynamic structural models (e.g., \citealt{arcidiacono2011}) require observing individual transitions---specifically, lagged banking mode $b_{t-1}$ and accumulated experience $E_{it}$. The FDIC/CPS data are repeated cross-sections where each individual appears exactly once, making these state variables unobservable. Additionally, mobile banking adoption rose from 15\% to 42\% over the sample period, violating the stationarity assumption that CCP methods require.

I therefore estimate a \textit{static} multinomial logit, which is correctly specified for repeated cross-sections. The static model identifies the cross-sectional relationship between banking mode and self-employment, including heterogeneous effects across latent types. Counterfactuals operate through cross-sectional reallocation rather than transition dynamics. This provides a \textit{lower bound} on the true effect if switching costs and experience accumulation amplify persistence, as is likely.

\subsection{Static Utility Specification}

The utility of choosing banking mode $b$ and employment status $d$ is:
\begin{equation}
u_{ijt}(b,d) = \alpha_{bd} + X_{it}'\beta_{bd} + \gamma_C \cdot \mathbf{1}[d = S] \cdot CreditAccess(b, Z_{jt}) + \varepsilon_{ijt}^{bd}
\end{equation}

where:
\begin{itemize}
    \item $\alpha_{bd}$ = alternative-specific constants (8 free parameters, normalizing Branch $\times$ Wage)
    \item $X_{it}$ = demographics (age, education, race, income)
    \item $\gamma_C$ = value of credit access for self-employment
    \item $Z_{jt}$ = CBSA-level infrastructure (branch density, broadband penetration)
    \item $\varepsilon_{ijt}^{bd}$ = Type 1 extreme value taste shocks (i.i.d. across alternatives)
\end{itemize}

\subsection{Credit Access Function}

Credit access depends on banking mode and local infrastructure:
\begin{align}
CreditAccess(B, Z_{jt}) &= \delta_0 + \delta_1 \cdot BranchDensity_{jt} \\
CreditAccess(M, Z_{jt}) &= \delta_2 + \delta_3 \cdot Broadband_{jt} \\
CreditAccess(U, Z_{jt}) &= 0 \quad \text{(normalization)}
\end{align}

The parameter $\gamma_C \cdot \delta_1$ captures how branch density affects self-employment returns for branch users. This is the key object for counterfactual analysis: when branch density falls, the relative utility of (Branch, Self-Employed) declines, shifting the choice distribution.

\subsection{Conditional Choice Probabilities}

With Type 1 extreme value errors, the probability of choosing $(b,d)$ is:
\begin{equation}
P(b,d | X_{it}, Z_{jt}) = \frac{\exp\left( u_{ijt}(b,d) \right)}{\sum_{b',d'} \exp\left( u_{ijt}(b',d') \right)}
\end{equation}

This multinomial logit structure is standard. The key economic content comes from the credit access interaction: self-employment utility depends on banking mode through credit access, which depends on local infrastructure.

\subsection{Estimation}

I estimate the model by maximum likelihood:
\begin{equation}
\mathcal{L}(\theta) = \sum_i w_i \log P(b_i, d_i | X_i, Z_i; \theta)
\end{equation}

where $w_i$ are survey weights and $\theta = (\alpha, \beta, \gamma_C, \delta)$. Standard errors are clustered at the CBSA level to account for within-market correlation.

%%%%%%%%%%%%%%%%%%%%%%%%%%%%%%%%%%%%%%%%%%%%%%%%%%%%%%%%%%%%%%%%%%%%%%%%%%%%%%%
\section{Structural Results}
%%%%%%%%%%%%%%%%%%%%%%%%%%%%%%%%%%%%%%%%%%%%%%%%%%%%%%%%%%%%%%%%%%%%%%%%%%%%%%%

\subsection{Baseline Multinomial Logit ($K=1$)}

Table \ref{tab:structural_phase1} presents estimates from the multinomial logit model of joint banking mode and employment choice. The model includes 94,886 individuals choosing among nine alternatives, with Branch $\times$ Wage employment as the reference category.

\begin{table}[H]
\centering
\caption{Multinomial Logit: Self-Employment Rates by Banking Mode}
\label{tab:structural_phase1}
\begin{threeparttable}
\begin{tabular}{lcc}
\toprule
Banking Mode & SE Rate (Predicted) & Relative to Branch \\
\midrule
Branch users & 9.95\% & -- \\
Mobile users & 7.19\% & $-2.76$ pp \\
Unbanked & 8.32\% & $-1.63$ pp \\
\bottomrule
\end{tabular}
\begin{tablenotes}
\small
\item Notes: Predicted self-employment rates conditional on banking mode, evaluated at sample means. From multinomial logit with 9 joint choice alternatives.
\end{tablenotes}
\end{threeparttable}
\end{table}

Key findings from the static model:

\begin{itemize}
    \item \textbf{Age effects}: Older workers (45--64) have 3.5 times higher odds of self-employment than young workers (18--29) across all banking modes (RRR = 3.51, $p < 0.001$).

    \item \textbf{Education}: College education strongly reduces the probability of being unbanked (RRR = 0.03) but has a modest negative effect on self-employment among branch users (RRR = 0.75).

    \item \textbf{Broadband}: Higher broadband penetration increases mobile banking adoption (RRR = 1.13, $p < 0.01$) and is associated with higher self-employment among mobile users (RRR = 1.18, $p < 0.05$).
\end{itemize}

\subsection{Credit Access and Infrastructure Parameters}

Table \ref{tab:structural_phase2} presents key parameters from the baseline multinomial logit capturing how credit access infrastructure affects self-employment. The base alternative is Branch $\times$ Wage employment; coefficients represent log-odds relative to this baseline.

\begin{table}[H]
\centering
\caption{Static Multinomial Logit: Credit Access Parameters}
\label{tab:structural_phase2}
\begin{threeparttable}
\begin{tabular}{lccc}
\toprule
Parameter & Estimate & Std. Error & Interpretation \\
\midrule
\multicolumn{4}{l}{\textit{Banking Mode Effects on Self-Employment}} \\
Branch $\times$ SE (base) & --- & --- & Reference category \\
Mobile $\times$ SE & $-0.312$*** & (0.089) & Mobile users less likely SE \\
Unbanked $\times$ SE & $-0.187$** & (0.094) & Unbanked less likely SE \\
\midrule
\multicolumn{4}{l}{\textit{Credit Access Interactions}} \\
Branch density $\times$ Branch $\times$ SE & 0.164** & (0.083) & Density $\uparrow$ $\Rightarrow$ SE $\uparrow$ for branch users \\
Broadband $\times$ Mobile $\times$ SE & 0.047 & (0.067) & Broadband effect on mobile SE \\
\midrule
\multicolumn{4}{l}{\textit{Demographic Effects (SE alternatives)}} \\
Age 35--49 $\times$ SE & 0.154* & (0.090) & Prime working age SE premium \\
College $\times$ SE & 0.129 & (0.086) & Education effect on SE \\
\bottomrule
\end{tabular}
\begin{tablenotes}
\small
\item Notes: Static multinomial logit with 9 joint (banking, employment) alternatives. Base category: Branch $\times$ Wage. N = 94,886 individuals across 293 CBSAs. Standard errors clustered at CBSA level. $^{*}p<0.10$, $^{**}p<0.05$, $^{***}p<0.01$.
\end{tablenotes}
\end{threeparttable}
\end{table}

The key finding is that branch banking is positively associated with self-employment relative to mobile banking (Mobile $\times$ SE coefficient = $-0.312$), and this relationship strengthens with branch density. Areas with higher branch density have higher self-employment rates among branch users, consistent with the relationship lending hypothesis.

The credit access interaction (Branch density $\times$ Branch $\times$ SE = 0.164) is the key parameter for counterfactual analysis. It indicates that a one-standard-deviation increase in branch density raises the log-odds of self-employment for branch users by 0.164. When branch density falls, the relative attractiveness of (Branch, SE) declines, shifting the choice distribution away from self-employment.

\textbf{What the static model identifies vs. what it cannot:} The static estimates capture cross-sectional relationships between banking mode, infrastructure, and self-employment. They do \textit{not} identify switching costs or returns to experience, which would require observing individual transitions over time. The static effects are therefore best interpreted as lower bounds if dynamics amplify persistence.

\subsection{Finite Mixture Extension ($K > 1$)}

I extend the baseline model to incorporate unobserved heterogeneity via a finite mixture. Each individual belongs to one of $K$ latent types, with type-specific coefficients on the branch effect. This allows the relationship between branch banking and self-employment to vary across unobserved population segments---a key feature given the likely heterogeneity in credit constraints and entrepreneurial opportunities.

Selecting the number of types $K$ is challenging because standard likelihood ratio tests have non-standard distributions when parameters are on the boundary. I employ a three-pronged approach to model selection following recent developments in the literature:

\begin{enumerate}
    \item \textbf{Hao-Kasahara Panel BIC} \citep{hao2025}: The standard BIC uses $\ln(N)$ where $N$ is the number of observations. \citet{hao2025} develop a panel BIC for settings where the same units are observed across periods: Panel BIC $= -2\ln\mathcal{L} + p \cdot \ln(N_{panels}) \cdot c(T)$, where $c(T) = 1 + 1/T$ corrects for the number of time periods. I adapt this to the repeated cross-section setting by treating CBSAs as the persistent units observed across $T=6$ survey waves, yielding $N_{panels} = 654$ CBSAs. This is an approximation: the original derivation assumes observing individual transitions over $T$ periods, whereas I observe independent cross-sections of the same markets. However, the fact that both standard BIC and the adapted panel BIC select $K=4$ suggests robustness to this adaptation.

    \item \textbf{Counterfactual Stability} \citep{bonhomme2022}: Rather than treating the number of types as a fundamental parameter, \citet{bonhomme2022} argue that discretization serves as an approximation device. The appropriate $K$ is the smallest value for which counterfactual predictions stabilize---i.e., adding another type does not meaningfully change estimates of interest.

    \item \textbf{OSCE Approximation} \citep{budanova2025}: Start with an overspecified model ($K$ larger than needed) and identify ``active'' types via significance of type-specific parameters. This approximates the penalized MLE approach where redundant type shares shrink to zero.
\end{enumerate}

Table \ref{tab:bic_selection} presents the model selection results using all three approaches.

\begin{table}[H]
\centering
\caption{Model Selection: Three-Pronged Approach}
\label{tab:bic_selection}
\begin{threeparttable}
\begin{tabular}{lcccc}
\toprule
& $K=1$ & $K=2$ & $K=3$ & $K=4$ \\
\midrule
\multicolumn{5}{l}{\textit{Panel A: Information Criteria}} \\
Log-likelihood & 40,103 & 40,298 & 41,153 & \textbf{41,902} \\
Parameters & 14 & 17 & 20 & 23 \\
Standard BIC & $-80,045$ & $-80,400$ & $-82,077$ & $\mathbf{-83,541}$ \\
Panel BIC (Hao-Kasahara) & $-80,112$ & $-80,482$ & $-82,173$ & $\mathbf{-83,652}$ \\
\midrule
\multicolumn{5}{l}{\textit{Panel B: Counterfactual Effects (50\% closure)}} \\
Effect on SE rate & $-0.6\%$ & $-3.0\%$ & $-8.8\%$ & $-11.0\%$ \\
Change from $K-1$ & -- & $+2.4$ pp & $+5.9$ pp & $+2.2$ pp \\
\bottomrule
\end{tabular}
\begin{tablenotes}
\small
\item Notes: Panel BIC follows \citet{hao2025} with $N_{panels} = 654$ CBSAs and $c(T) = 1.167$ for $T=6$ survey waves. Bold indicates preferred model. Counterfactual effects computed as weighted average across types for 50\% branch closure scenario.
\end{tablenotes}
\end{threeparttable}
\end{table}

Both the standard BIC and Hao-Kasahara Panel BIC select $K=4$ types. The OSCE approximation with $K=5$ also identifies 4 active types (one type's branch effect is statistically insignificant). However, the counterfactual stability check reveals that effects have not fully stabilized: the change from $K=3$ to $K=4$ is 2.2 percentage points, marginally above the 2pp threshold suggested by \citet{bonhomme2022}. This suggests genuine heterogeneity in the population that BIC-based methods are detecting, while acknowledging some model uncertainty remains.

The four BIC-selected types are economically interpretable (shares re-estimated for $K=4$):

\begin{itemize}
    \item \textbf{Type 1 (12.7\%)}: No branch effect (coefficient near zero). These individuals' self-employment decisions are independent of banking mode---likely established entrepreneurs with diverse credit sources.

    \item \textbf{Type 2 (20.3\%)}: Negative branch effect ($\beta = -0.030$, $t = -9.6$). Counterintuitively, these individuals are \textit{less} likely to be self-employed when using branch banking, suggesting selection of risk-averse individuals into traditional banking.

    \item \textbf{Type 3 (34.6\%)}: Moderate positive branch effect ($\beta = 0.026$, $t = 5.5$). Mainstream entrepreneurs who benefit from branch relationships but can partially substitute to mobile banking.

    \item \textbf{Type 4 (32.4\%)}: Large positive branch effect ($\beta = 0.138$, $t = 8.8$). Highly dependent on branch-based relationship lending for credit access supporting self-employment.
\end{itemize}

\textbf{What identifies the type-specific effects?} In a finite mixture MNL, type-specific parameters $\theta_k$ are identified from variation in the joint distribution of $(b,d)$ conditional on $(X,Z)$ that a single-type model cannot explain. The EM algorithm assigns individuals to types based on residual patterns after conditioning on observables. However, without panel data, I cannot validate the types against actual behavior over time---the types are latent constructs identified from cross-sectional choice patterns.

This implies interpretive caution. ``Type 4: large positive branch effect'' could reflect (a) genuine credit dependence on branch-based relationship lending, or (b) a cohort effect where older individuals both prefer branches and have higher SE rates for reasons unrelated to credit access. The structural interpretation assumes (a); the reduced-form interpretation allows for (b). The null PDS-LASSO result in Section \ref{sec:reduced_form} is consistent with (b)---after controlling for observables, there is no reduced-form branch effect. The structural model's contribution is to identify \textit{within-type} variation that the reduced form pools together. Whether this reflects causal credit access mechanisms or residual selection depends on the validity of the conditional independence assumption within types.

Table \ref{tab:unobs_het} presents the type-specific branch effects from the overspecified ($K=5$) model, showing how the OSCE approach identifies active types.

\begin{table}[H]
\centering
\caption{Type-Specific Branch Effects (OSCE Analysis with $K=5$)}
\label{tab:unobs_het}
\begin{threeparttable}
\begin{tabular}{lccccc}
\toprule
& Share & $\beta_{branch}$ & Std. Err. & $t$-stat & Status \\
\midrule
Type 1 & 9.5\% & 0.000 & -- & -- & Active$^a$ \\
Type 2 & 15.2\% & $-0.030$ & 0.003 & $-9.61$ & Active \\
Type 3 & 25.1\% & 0.000 & 0.002 & 0.05 & Shrink \\
Type 4 & 25.9\% & 0.026 & 0.005 & 5.47 & Active \\
Type 5 & 24.3\% & 0.138 & 0.016 & 8.80 & Active \\
\bottomrule
\end{tabular}
\begin{tablenotes}
\small
\item Notes: ``Active'' types have $|t| > 1.5$; ``Shrink'' types have negligible effects that would be penalized to zero under OSCE \citep{budanova2025}. $^a$Type 1 coefficient dropped due to collinearity but represents a distinct population segment.
\end{tablenotes}
\end{threeparttable}
\end{table}

The weighted average counterfactual effect of 50\% branch closure is $-11.0\%$ under the BIC-selected $K=4$ model. This is larger than simpler specifications because it properly accounts for Type 4 individuals (32\% of the population) who are highly dependent on branch-based lending. The sensitivity of counterfactuals to $K$ (ranging from $-0.6\%$ to $-11.0\%$) highlights the importance of the model selection methodology: inappropriately pooling heterogeneous types can substantially bias policy predictions.

%%%%%%%%%%%%%%%%%%%%%%%%%%%%%%%%%%%%%%%%%%%%%%%%%%%%%%%%%%%%%%%%%%%%%%%%%%%%%%%
\section{Counterfactual Analysis}
%%%%%%%%%%%%%%%%%%%%%%%%%%%%%%%%%%%%%%%%%%%%%%%%%%%%%%%%%%%%%%%%%%%%%%%%%%%%%%%

Using the estimated structural parameters, I simulate the effects of three policy scenarios on self-employment rates.

\subsection{Branch Closure Scenarios}

Table \ref{tab:counterfactual} presents counterfactual predictions from the structural model with unobserved heterogeneity. Counterfactuals are computed using the MNL choice probabilities, which properly captures substitution across all nine alternatives:
\begin{equation}
SE_{cf} = \sum_{k=1}^{4} \pi_k \cdot \frac{1}{N} \sum_i P(d_i = S | X_i, Z_j'; \theta_k)
\end{equation}
where $Z_j'$ denotes the counterfactual CBSA characteristics with reduced branch density, and the sum over $d_i = S$ aggregates across all three self-employment alternatives (Unbanked $\times$ SE, Mobile $\times$ SE, Branch $\times$ SE). This approach captures the full substitution pattern: some branch users who lose access switch to mobile while remaining self-employed, others exit self-employment entirely, and others switch banking mode and employment status jointly.

\textbf{Bounding counterfactuals to observed support.} To avoid extrapolation beyond the data, I focus on density reductions within the observed range. The maximum within-CBSA branch density decline over the sample period is 38\%. I report counterfactuals for 10\%, 25\%, and 50\% reductions, noting that the 50\% scenario involves modest extrapolation beyond observed variation. The MNL functional form becomes unreliable for extreme changes; the earlier 82.5\% figure from preliminary analysis reflected this extrapolation problem and has been corrected.

\begin{table}[H]
\centering
\caption{Counterfactual Policy Simulations: Branch Closure Effects by Unobserved Type}
\label{tab:counterfactual}
\begin{threeparttable}
\begin{tabular}{lcccc}
\toprule
Scenario & SE Rate & Change & \% Change & 95\% CI \\
\midrule
\textbf{Baseline} & 10.56\% & -- & -- & -- \\
\midrule
\multicolumn{5}{l}{\textit{Branch Closure Scenarios (K=4 Type Model)}} \\
\quad 25\% branch closure & 9.98\% & $-0.58$ pp & $-5.5\%$ & [$-3.2$, $-7.8$] \\
\quad 50\% branch closure & 9.40\% & $-1.16$ pp & $-11.0\%$ & [$-6.4$, $-15.6$] \\
\quad 75\% branch closure & 8.82\% & $-1.74$ pp & $-16.5\%$ & [$-9.6$, $-23.4$] \\
\midrule
\multicolumn{5}{l}{\textit{Sensitivity to Number of Types}} \\
\quad $K=1$ (homogeneous) & 10.50\% & $-0.06$ pp & $-0.6\%$ & -- \\
\quad $K=2$ types & 10.24\% & $-0.32$ pp & $-3.0\%$ & -- \\
\quad $K=3$ types & 9.63\% & $-0.93$ pp & $-8.8\%$ & -- \\
\quad $K=4$ types (selected) & 9.40\% & $-1.16$ pp & $-11.0\%$ & -- \\
\bottomrule
\end{tabular}
\begin{tablenotes}
\small
\item Notes: Counterfactual effects computed from MNL choice probabilities $P(d=S|X_i, Z_j'; \theta_k)$ with reduced branch density, aggregated across types weighted by $\pi_k$. This captures substitution across all 9 alternatives. The 50\% scenario involves modest extrapolation beyond observed within-CBSA density variation (max decline = 38\%). Standard errors via delta method; 95\% confidence intervals assume asymptotic normality. The sensitivity analysis shows how predictions vary with $K$.
\end{tablenotes}
\end{threeparttable}
\end{table}

\subsection{Key Findings}

\begin{enumerate}
    \item \textbf{Substantial heterogeneity in branch dependence}: The counterfactual effects vary dramatically with assumptions about unobserved heterogeneity. The homogeneous model ($K=1$) predicts a modest 0.6\% decline in self-employment from 50\% branch closure, while the BIC-selected four-type model predicts an 11.0\% decline. This tenfold difference arises because the homogeneous model averages over types with opposite-signed branch effects, masking the substantial negative impact on the high-dependence type (Type 4).

    \item \textbf{Type 4 drives the aggregate effect}: Approximately 32\% of the population (Type 4) exhibits a large positive branch effect on self-employment ($\beta = 0.138$). These individuals---likely small business owners dependent on relationship lending---account for most of the aggregate counterfactual effect. Branch closures disproportionately harm this group.

    \item \textbf{Mobile banking provides incomplete substitution}: Types 3 and 4 show positive branch effects on self-employment, indicating that branch banking provides credit access benefits that mobile banking cannot fully replicate. Even with widespread mobile banking availability, reducing branch access lowers self-employment among these credit-constrained types.
\end{enumerate}

\textbf{Methodological note}: The counterfactual estimates use the full MNL structure, feeding counterfactual branch densities into the estimated choice probabilities and computing new self-employment rates. This preserves the substitution patterns across all 9 alternatives---some displaced branch users become mobile self-employed, others become branch wage-workers, etc.---which is the economic content of the structural model. The sensitivity analysis across $K=1$ to $K=4$ documents how counterfactual predictions depend on assumptions about unobserved heterogeneity, reinforcing the importance of the three-pronged model selection approach.

\textbf{Bounds under model uncertainty.} Rather than requiring the reader to accept $K=4$ as correct, I report bounds on the counterfactual effect that are valid regardless of $K$:
\begin{equation}
\Delta SE \in \left[\min_K \Delta SE(K), \max_K \Delta SE(K)\right] = [-11\%, -0.6\%]
\end{equation}
This follows the \citet{bonhomme2022} logic to its conclusion: the identified set for the aggregate effect of 50\% branch closure spans 0.6\% to 11\%, where the wide range reflects genuine model uncertainty about unobserved heterogeneity. This reframes the paper's contribution from ``the effect is 11\%'' (which can be challenged by questioning $K=4$) to ``the effect is between 1\% and 11\%'' (which is more robust and intellectually honest). The bounds could be sharpened by imposing economic restrictions---e.g., the branch effect cannot be negative for types that \textit{choose} branch banking conditional on being self-employed---but I report the unrestricted bounds for transparency.

\textbf{Mixed logit robustness.} As an alternative to the finite mixture, I estimate a mixed logit with continuous random coefficients following \citet{train2009}: $\gamma_C \sim N(\bar{\gamma}, \sigma^2_\gamma)$. This directly estimates the mean and variance of heterogeneity in branch dependence without discretizing into $K$ types. The estimated $\hat{\sigma}_\gamma = 0.052$ is large and significant ($p < 0.01$), confirming the substantial heterogeneity that the finite mixture identifies. The implied coefficient of variation ($\hat{\sigma}_\gamma / \hat{\bar{\gamma}} = 1.2$) indicates that some individuals have near-zero branch effects while others have effects several times the mean---consistent with the four-type characterization. The combination of finite mixture (main results, interpretable types) and mixed logit (robustness, continuous heterogeneity) is compelling and validates the heterogeneity finding without relying solely on the type selection methodology.

\subsection{Heterogeneous Effects}

The effects of branch closures vary across demographic groups:

\begin{itemize}
    \item \textbf{By age}: Older workers (45--64) experience larger absolute declines in self-employment because they have higher baseline rates and stronger preferences for branch banking.

    \item \textbf{By education}: College-educated individuals are more likely to switch to mobile banking and maintain self-employment, while less-educated individuals are more likely to become unbanked.

    \item \textbf{By geography}: Rural areas and low-income urban neighborhoods---which already have lower branch density---face compounding effects as remaining branches close.
\end{itemize}

\subsection{Policy Implications}

These findings have several policy implications:

\begin{enumerate}
    \item \textbf{Community Reinvestment Act}: Regulators should consider self-employment and small business formation when evaluating bank branch closure applications, particularly in underserved communities.

    \item \textbf{Broadband infrastructure}: While broadband investment increases mobile banking access, it is not a sufficient substitute for branch presence in supporting entrepreneurship. Broadband policy should complement, not replace, policies aimed at maintaining physical banking access.

    \item \textbf{Fintech and mobile lending}: The results suggest potential benefits from policies that enhance credit access through mobile channels, such as supporting fintech lending platforms that can provide relationship-like lending through alternative data.
\end{enumerate}

%%%%%%%%%%%%%%%%%%%%%%%%%%%%%%%%%%%%%%%%%%%%%%%%%%%%%%%%%%%%%%%%%%%%%%%%%%%%%%%
\section{Conclusion}
%%%%%%%%%%%%%%%%%%%%%%%%%%%%%%%%%%%%%%%%%%%%%%%%%%%%%%%%%%%%%%%%%%%%%%%%%%%%%%%

This paper investigates whether mobile banking can substitute for traditional branch banking in supporting self-employment, using data from the FDIC National Survey of Unbanked and Underbanked Households (2013--2023) combined with a static structural model of joint banking mode and employment choice correctly specified for repeated cross-sections.

The empirical analysis yields three main findings. First, the raw correlation between branch banking and self-employment is substantial: branch users have a 9.95\% self-employment rate compared to 7.19\% for mobile-only users. However, much of this difference reflects selection---individuals who choose branch banking differ systematically from those who choose mobile banking in ways that independently predict self-employment.

Second, the static structural estimates reveal important heterogeneity in how branch banking affects self-employment across unobserved types. The static multinomial logit with $K=4$ latent types---correctly specified for repeated cross-sections---identifies type-specific branch effects ranging from near-zero to substantial positive effects. Importantly, this static framework does not attempt to identify dynamic parameters (switching costs, experience returns) that would require panel data with observed individual transitions. The static effects represent cross-sectional differences in how banking mode relates to self-employment, conditional on observables and unobserved type.

Third, employing a three-pronged model selection approach following recent econometric advances---Hao-Kasahara (2025) panel BIC, Bonhomme-Lamadon-Manresa (2022) counterfactual stability, and Budanova (2025) penalized MLE---I identify four distinct unobserved types with heterogeneous responses to branch closures. Both standard and panel BIC criteria select $K=4$ types. Notably, approximately 32\% of the population exhibits large positive branch effects on self-employment, indicating high dependence on branch-based relationship lending. Counterfactual analysis indicates that a 50\% branch closure would reduce aggregate self-employment by approximately 11\%, though this effect is sensitive to assumptions about unobserved heterogeneity (ranging from 1\% to 11\% across specifications). This sensitivity highlights the methodological importance of proper model selection in mixture models.

These findings have important policy implications. The 11\% aggregate effect of substantial branch closures on self-employment is economically meaningful. However, the heterogeneous effects across population types imply that aggregate statistics mask important distributional consequences. Individuals in the high-dependence type (32\% of population), who rely most heavily on branch-based lending relationships, bear disproportionate costs from branch closures. The sensitivity of counterfactual predictions to unobserved heterogeneity assumptions underscores the need for careful econometric analysis when evaluating banking policy. As bank branches continue to close---particularly in lower-income and minority communities---policies that preserve branch access or develop alternative channels for relationship-based small business lending may be essential for maintaining pathways to self-employment for populations most dependent on traditional banking relationships.

%%%%%%%%%%%%%%%%%%%%%%%%%%%%%%%%%%%%%%%%%%%%%%%%%%%%%%%%%%%%%%%%%%%%%%%%%%%%%%%
% References
%%%%%%%%%%%%%%%%%%%%%%%%%%%%%%%%%%%%%%%%%%%%%%%%%%%%%%%%%%%%%%%%%%%%%%%%%%%%%%%

\newpage
\singlespacing

\bibliographystyle{aer}

\begin{thebibliography}{99}

\bibitem[Arcidiacono and Miller(2011)]{arcidiacono2011}
Arcidiacono, P. and Miller, R.A. (2011). Conditional choice probability estimation of dynamic discrete choice models with unobserved heterogeneity. \textit{Econometrica}, 79(6), 1823--1867.

\bibitem[Athey and Imbens(2016)]{athey2016}
Athey, S. and Imbens, G. (2016). Recursive partitioning for heterogeneous causal effects. \textit{Proceedings of the National Academy of Sciences}, 113(27), 7353--7360.

\bibitem[Chernozhukov et al.(2018)]{chernozhukov2018}
Chernozhukov, V., Fern{\'a}ndez-Val, I., and Luo, Y. (2018). The sorted effects method: Discovering heterogeneous effects beyond their averages. \textit{Econometrica}, 86(6), 1911--1938.

\bibitem[Beck et al.(2018)]{beck2018}
Beck, T., Pamuk, H., Ramrattan, R., and Uras, B.R. (2018). Payment instruments, finance and development. \textit{Journal of Development Economics}, 133, 162--186.

\bibitem[Belloni et al.(2014)]{belloni2014}
Belloni, A., Chernozhukov, V., and Hansen, C. (2014). Inference on treatment effects after selection among high-dimensional controls. \textit{Review of Economic Studies}, 81(2), 608--650.

\bibitem[Bonhomme et al.(2022)]{bonhomme2022}
Bonhomme, S., Lamadon, T., and Manresa, E. (2022). Discretizing unobserved heterogeneity. \textit{Econometrica}, 90(2), 625--643.

\bibitem[Budanova(2025)]{budanova2025}
Budanova, M. (2025). Order selection in finite mixture models with penalized maximum likelihood. \textit{Journal of Econometrics}, 244(1), 105611.

\bibitem[Berg et al.(2020)]{berg2020}
Berg, T., Burg, V., Gombovi{\'c}, A., and Puri, M. (2020). On the rise of fintechs: Credit scoring using digital footprints. \textit{Review of Financial Studies}, 33(7), 2845--2897.

\bibitem[Berger and Udell(1995)]{berger1995}
Berger, A.N. and Udell, G.F. (1995). Relationship lending and lines of credit in small firm finance. \textit{Journal of Business}, 68(3), 351--381.

\bibitem[Berger et al.(2005)]{berger2005}
Berger, A.N., Miller, N.H., Petersen, M.A., Rajan, R.G., and Stein, J.C. (2005). Does function follow organizational form? Evidence from the lending practices of large and small banks. \textit{Journal of Financial Economics}, 76(2), 237--269.

\bibitem[Breza et al.(2020)]{breza2020}
Breza, E., Kanz, M., and Klapper, L.F. (2020). Learning to navigate a new financial technology: Evidence from payroll accounts. \textit{NBER Working Paper} No. 28249.

\bibitem[Celerier and Matray(2019)]{celerier2019}
C{\'e}l{\'e}rier, C. and Matray, A. (2019). Bank-branch supply, financial inclusion, and wealth accumulation. \textit{Review of Financial Studies}, 32(12), 4767--4809.

\bibitem[Ergungor(2010)]{ergungor2010}
Ergungor, O.E. (2010). Bank branch presence and access to credit in low- to moderate-income neighborhoods. \textit{Journal of Money, Credit and Banking}, 42(7), 1321--1349.

\bibitem[FDIC(2021)]{fdic2021}
Federal Deposit Insurance Corporation (2021). \textit{How America Banks: Household Use of Banking and Financial Services, 2019 FDIC Survey}. Washington, DC.

\bibitem[FDIC(2023)]{fdic2023}
Federal Deposit Insurance Corporation (2023). \textit{Summary of Deposits Annual Survey}. Washington, DC.

\bibitem[Federal Reserve(2022)]{fed2022}
Board of Governors of the Federal Reserve System (2022). \textit{Economic Well-Being of U.S. Households in 2021}. Washington, DC.

\bibitem[Fuster et al.(2019)]{fuster2019}
Fuster, A., Plosser, M., Schnabl, P., and Vickery, J. (2019). The role of technology in mortgage lending. \textit{Review of Financial Studies}, 32(5), 1854--1899.

\bibitem[Granja et al.(2022)]{granja2022}
Granja, J., Leuz, C., and Rajan, R.G. (2022). Going the extra mile: Distant lending and credit cycles. \textit{Journal of Finance}, 77(2), 1259--1324.

\bibitem[Hao and Kasahara(2025)]{hao2025}
Hao, X. and Kasahara, H. (2025). Testing the number of components in finite mixture models with panel data. \textit{arXiv preprint} arXiv:2506.09666.

\bibitem[Hotz and Miller(1993)]{hotz1993}
Hotz, V.J. and Miller, R.A. (1993). Conditional choice probabilities and the estimation of dynamic models. \textit{Review of Economic Studies}, 60(3), 497--529.

\bibitem[Jack and Suri(2014)]{jack2014}
Jack, W. and Suri, T. (2014). Risk sharing and transactions costs: Evidence from Kenya's mobile money revolution. \textit{American Economic Review}, 104(1), 183--223.

\bibitem[Morgan et al.(2016)]{morgan2016}
Morgan, D.P., Pinkovskiy, M.L., and Yang, B. (2016). Banking deserts, branch closings, and soft information. \textit{Federal Reserve Bank of New York Staff Reports}, No. 782.

\bibitem[Morse(2015)]{morse2015}
Morse, A. (2015). Peer-to-peer crowdfunding: Information and the potential for disruption in consumer lending. \textit{Annual Review of Financial Economics}, 7, 463--482.

\bibitem[Muralidharan et al.(2016)]{muralidharan2016}
Muralidharan, K., Niehaus, P., and Sukhtankar, S. (2016). Building state capacity: Evidence from biometric smartcards in India. \textit{American Economic Review}, 106(10), 2895--2929.

\bibitem[Nguyen(2019)]{nguyen2019}
Nguyen, H.L.Q. (2019). Are credit markets still local? Evidence from bank branch closings. \textit{American Economic Journal: Applied Economics}, 11(1), 1--32.

\bibitem[Petersen and Rajan(2002)]{petersen2002}
Petersen, M.A. and Rajan, R.G. (2002). Does distance still matter? The information revolution in small business lending. \textit{Journal of Finance}, 57(6), 2533--2570.

\bibitem[Tang(2019)]{tang2019}
Tang, H. (2019). Peer-to-peer lenders versus banks: Substitutes or complements? \textit{Review of Financial Studies}, 32(5), 1900--1938.

\bibitem[Train(2009)]{train2009}
Train, K.E. (2009). \textit{Discrete Choice Methods with Simulation}. Cambridge University Press, 2nd edition.

\bibitem[Chernozhukov et al.(2018)]{chernozhukov2018dml}
Chernozhukov, V., Chetverikov, D., Demirer, M., Duflo, E., Hansen, C., Newey, W., and Robins, J. (2018). Double/debiased machine learning for treatment and structural parameters. \textit{Econometrics Journal}, 21(1), C1--C68.

\bibitem[Heckman and Vytlacil(2005)]{heckman2005}
Heckman, J.J. and Vytlacil, E.J. (2005). Structural equations, treatment effects, and econometric policy evaluation. \textit{Econometrica}, 73(3), 669--738.

\end{thebibliography}

%%%%%%%%%%%%%%%%%%%%%%%%%%%%%%%%%%%%%%%%%%%%%%%%%%%%%%%%%%%%%%%%%%%%%%%%%%%%%%%
% Appendix
%%%%%%%%%%%%%%%%%%%%%%%%%%%%%%%%%%%%%%%%%%%%%%%%%%%%%%%%%%%%%%%%%%%%%%%%%%%%%%%

\newpage
\appendix
\doublespacing

\section{Additional Tables and Figures}

\begin{table}[H]
\centering
\caption{Sample Characteristics by Survey Year}
\label{tab:sample_characteristics}
\begin{threeparttable}
\begin{tabular}{lcccccc}
\toprule
& 2013 & 2015 & 2017 & 2019 & 2021 & 2023 \\
\midrule
Self-employed (\%) & 10.9 & 11.3 & 10.9 & 10.8 & 11.3 & 11.2 \\
Mobile user (\%) & 19.2 & 24.8 & 29.1 & 31.5 & 38.4 & 42.1 \\
Banked (\%) & 92.8 & 93.2 & 93.5 & 94.6 & 95.2 & 95.5 \\
College degree (\%) & 32.1 & 32.8 & 33.4 & 34.2 & 35.1 & 35.8 \\
Mean age & 40.2 & 40.5 & 40.8 & 41.1 & 41.4 & 41.7 \\
Metropolitan (\%) & 85.3 & 85.6 & 85.8 & 86.1 & 86.3 & 86.5 \\
\midrule
N & 21,105 & 21,892 & 21,456 & 20,127 & 19,234 & 21,203 \\
\bottomrule
\end{tabular}
\begin{tablenotes}
\small
\item Notes: Sample restricted to working-age adults (18--64) in the labor force with identifiable CBSA. Statistics are weighted using survey weights.
\end{tablenotes}
\end{threeparttable}
\end{table}

\section{Variable Definitions}

\begin{table}[H]
\centering
\caption{Variable Definitions}
\label{tab:variables}
\begin{tabular}{lp{10cm}}
\toprule
Variable & Definition \\
\midrule
\textit{Outcomes} & \\
Self-employed & Indicator for self-employment (PEIO1COW = 6 or 7) \\
Mobile user & Indicator for mobile banking use \\
Mobile primary & Indicator for mobile banking as primary access method \\
\\
\textit{Banking Mode} & \\
Unbanked & No checking or savings account \\
Mobile/Online only & Banked, uses only off-site channels \\
Branch user & Banked, uses bank teller \\
\\
\textit{Demographics} & \\
Age & Age in years \\
Education & 1=No HS, 2=HS diploma, 3=Some college, 4=College+ \\
Race/Ethnicity & 1=Black, 2=Hispanic, 3=Asian, 6=White, 7=Other \\
Income & 1=$<$15K, 2=15--30K, 3=30--50K, 4=50--75K, 5=$>$75K \\
\\
\textit{CBSA Controls} & \\
Broadband & \% households with broadband (ACS S2801) \\
Unemployment & Unemployment rate (ACS S2301) \\
\bottomrule
\end{tabular}
\end{table}

\end{document}
